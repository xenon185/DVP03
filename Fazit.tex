\section{Fazit}
%\begin{itemize}
%	\item Phase: Linearer abfallender Verlauf Zwischen zwei Polstellen. An den Polstellen sind Phasensprünge zu beobachten.
%\end{itemize}
\noindent Während der Labordurchführung ist bewusst geworden, dass es zwingend notwendig ist, den Frequenzgang des DSK-Boards bei den Messungen zu berücksichtigen. Ohne dieses Wissen stünde man vor großen Rätseln, weshalb der Frequenzgang eines Hochpasses im Durchlassbereich bei $f>3.5kHz$ abfällt. Dem DSK-Boards ist eingangsseitig, abhängig der Abtastfrequenz $f_s$, ein Anti-Aliasingfilter (Tiefpass) vorgeschaltet, um eben Anti-Aliasingeffekte bei Frequenzen $>\frac{f_s}{2}$ zu vermeiden.\\
\noindent Ebenso ist zu beachten, dass die Eckfrequenz eines digitalen Filters nicht wie bei einem analogen Filter bei -3dB unter dem Maximum liegt, sondern bei -6dB unter dem Maximum.\\
\noindent In der Vorbereitung haben wir fälschlicherweise mit einer zu niedrigen Filterordnung (N = 16) und somit mit einer zu geringen Anzahl an Koeffizienten (Anzahl = 17) gerechnet. Die Vorgabe der Sperrdämpfung (-40dB) konnte somit weder bei MATLAB noch auf dem DSP eingehalten werden. Ein direkter Vergleich ergab keine signifikanten Abweichungen, wider erwarten. Eine nachfolgende Berechnung der benötigten Filter Ordnung ergab N = 18. Die Anzahl der Koeffizienten ergibt sich daraus zu 19. Wir erwarten bei höheren Ordnungen auch höhere Abweichungen im direkten Vergleich zwischen der MATLAB-Berechnung und dem DSP. Deshalb ist es zwingend notwendig, die theoretisch Berechneten Frequenzgänge der entwickelten Filter ebenso auf dem DSP zu implementieren und zu testen. Man würde den Einfluss des Frequenzganges des DSPs und die gegebenenfalls vorhandenen Rundungsfehler des Boards erkennen und anschließend berücksichtigen.\\
\noindent Bei der Berechnung des FIR-Filters mit den gewünschten Eigenschaften muss folgender Zusammenhang gelten: Anzahl Filterkoeffizienten = 1 + Filterordnung. Eine Transformation eines vorhandenen Tiefpasses in andere Filtertypen ist nur möglich, wenn das vorhandene FIR-Tiefpassfilter eine ungeradzahlige Koeffizientenanzahl, das heißt eine geradzahlige Filterordnung besitzt.\\
\noindent Wir haben durch ein von uns durchgeführtes Profiling des FIR-Tiefpasses eine wichtige Erkenntnis gewinnen können. Die Frequenz, mit der die ISR abgearbeitet wird, beträgt etwas über 165kHz. Mit einer genutzten Abtastrate von 8kHz befinden wir uns deutlich unter dieser Grenze. Die Abtastfrequenz könnte theoretisch bis 165kHz hochgedreht werden, sodass der nächste Abtastwert noch korrekt eingelesen wird und kein Aliasing entsteht. Die physischen Eigenschaften des DSPs wurde mit unserer gewählten Abtastfrequenz noch lange nicht ausgereizt, es ist noch viel Luft, bzw. Puffer nach oben hin vorhanden. Beim FIR-Filterentwurf ist die verwendete Methode (Fensterung) sehr entscheidet für die Filtercharakteristik. Beim Vergleich der mit Matlab berechneten Filterkoeffizienten sieht man sehr gut das die Funktion firpm gegenüber der fir1(Hamming-Window) eine geringer Filterordnung benötigt um auf die geforderten Eigenschaften zu kommen und dazu einen schmaleren Übergangsbereich aufweist.